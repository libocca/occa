\documentclass[12pt]{article}
\usepackage[lmargin=1in, rmargin=1in, bmargin=1in, tmargin=1in, papersize={8.50in,11.00in}]{geometry}

\usepackage{dsm5}

\setTitle{\OKL: A Unified Kernel Language for Parallel Architectures}

\addAuthor{
  David S. Medina\footnote{David S. Medina, Computational and Applied Mathematics at Rice University (dsm5@rice.edu)}
}

\addAuthor{
  Amik St-Cyr\footnote{Amik St-Cyr, HPC Development Lead  at Shell  (Amik.St-Cyr@shell.com)}
}

\addAuthor{
  Tim Warburton\footnote{Tim Warburton, Computational and Applied Mathematics at Rice University (timwar@rice.edu)}
}

\addAuthor{
  Lucas Wilcox\footnote{Lucas Wilcox, Department of Applied Mathematics at Naval Postgraduate School (lwilcox@nps.edu)}
}

\begin{document}

\pagestyle{empty}

\placeTitlePage

The inability to predict lasting languages and architectures led us to develop \occa, a library focused on host-device interaction.
\occa can be natively used with C, C++, C\#, Fortran, Python, Julia and MATLAB.
The unified kernel language in \occa is based on macro expansions exposing parallelism and expanding to OpenMP, OpenCL, CUDA, Pthreads and COI.

Rather than coding in the \occa macro-based language, we introduce two native languages: \okl and \ofl.
The \occa Kernel Language (\okl) is based on C and extends the language by exposing parallel loops and labeling them.
The \occa Fortran Language (\ofl) is the Fortran equivalent of \okl.
Simple examples of both, \okl and \ofl, can be seen in \refLst{addVectors} and \refLst{ofl_addVectors}.

The memory hierarchy seen in the GPU
With parsing tools available from developing \okl and \ofl, we added features
\okl and \ofl include features such as:
\begin{itemize}
\item Supporting the shared-memory model for GPU-architectures
\item Supporting multiple outer-loops per kernel
\item Automatic detection on work-group/work-item (block/thread) sizes
\item Auto-translates basic OpenCL/CUDA kernels to \okl/\ofl
\item Auto-barrier insertion between inner-loops
\end{itemize}

\setCode{OKL,none,\scriptsize}
\vspace{4mm}
\begin{lstlisting}[caption={Adding two vectors using \occa Kernel Language (\okl)},label={lst:addVectors}]
kernel void addVectors(int entries, const float *a, const float *b, float *ab){
  for(int group = 0; group < entries; group += 16; outer0){
    for(int item = 0; item < 16; ++item; inner0){
      const int n = (item + (16 * group));

      if(n < entries)
        ab[n] = a[n] + b[n];
    }
  }
}
\end{lstlisting}

\setCode{OFL,none,\scriptsize}
\vspace{4mm}
\begin{lstlisting}[caption={Adding two vectors using \occa Fortran Language (\ofl)},label={lst:ofl_addVectors}]
kernel subroutine addVectors(entries, a, b, ab)
  implicit none

  integer(4), intent(in)  :: entries
  real(4)   , intent(in)  :: a(:), b(:)
  real(4)   , intent(out) :: ab(:)

  integer :: group, item, n

  do group = 1, entries, 16, outer0
     do item = 1, 16, inner0
        n = (item + (16 * (group - 1)))

        if (n < entries) then
           ab(n) = a(n) + b(n)
        end if
     end do
  end do

end subroutine addVectors
\end{lstlisting}

\end{document}